\documentclass[10pt, oneside]{article} 
\usepackage{amsmath, amsthm, amssymb, calrsfs, wasysym, verbatim, bbm, color, graphics, geometry}

\geometry{tmargin=.75in, bmargin=.75in, lmargin=.75in, rmargin = .75in}  

\newcommand{\R}{\mathbb{R}}
\newcommand{\C}{\mathbb{C}}
\newcommand{\Z}{\mathbb{Z}}
\newcommand{\N}{\mathbb{N}}
\newcommand{\Q}{\mathbb{Q}}
\newcommand{\Cdot}{\boldsymbol{\cdot}}

\newtheorem{thm}{Theorem}
\newtheorem{defn}{Definition}
\newtheorem{conv}{Convention}
\newtheorem{rem}{Remark}
\newtheorem{lem}{Lemma}
\newtheorem{cor}{Corollary}


\title{Causual Inference Note}
\author{Kirby CHEN}
\date{Academic Year 2024-2025}

\begin{document}

\maketitle
\tableofcontents

\vspace{.25in}

\section{PSM Model Setup}

For an individual $i$, the outcome depends on whether they receive a certain treatment:
\begin{equation}
y_i =
\begin{cases}
y_{1i}, & \text{if } D_i = 1 \\
y_{0i}, & \text{if } D_i = 0
\end{cases}
\end{equation}

\begin{itemize}
    \item $D_i$ indicates whether individual $i$ receives the treatment, where $1$ represents treated, and $0$ represents untreated.
    \item $y_{1i}$ represents the outcome for individual $i$ if treated.
    \item $y_{0i}$ represents the outcome for individual $i$ if untreated.
\end{itemize}

Given the observable covariates $x_i$, the probability of an individual $i$ receiving the treatment is defined as:
\begin{equation}
p(x_i) = \Pr(D_i = 1 \mid x = x_i) = E(D_i \mid x_i)
\end{equation}

Based on Equations (1) and (2), the **Average Treatment Effect on the Treated (ATT)** is given by:
\begin{equation}
ATT = E[y_{1i} - y_{0i} \mid D_i = 1]
\end{equation}
\begin{equation}
= E[E[y_{1i} - y_{0i} \mid D_i = 1, p(x_i)]]
\end{equation}
\begin{equation}
= E[E[y_{1i} \mid D_i = 1, p(x_i)] - E[y_{0i} \mid D_i = 0, p(x_i)] \mid D_i = 1]
\end{equation}

\subsection{Definition of ATE}
The treatment effect is defined as a random variable:
\begin{equation}
y_{1i} - y_{0i}
\end{equation}
The expected value of this effect is called the **Average Treatment Effect (ATE)**:
\begin{equation}
ATE = E(y_{1i} - y_{0i})
\end{equation}
ATE represents the expected treatment effect for a randomly selected individual from the population, regardless of whether they received the treatment.

\subsection{Definition of ATT}
If we consider only those who actually participated in the treatment, we define the **Average Treatment Effect on the Treated (ATT)**:
\begin{equation}
ATT = E(y_{1i} - y_{0i} \mid D_i = 1)
\end{equation}
ATT measures the **average effect of the treatment for those who received it**. 

\subsection{Importance of ATT vs. ATE}
For policymakers, **ATT is often more important** because it directly measures the effect on those who received the intervention. However, ATE and ATT are generally **not equal**.

\subsection{Estimation Challenges and Selection Bias}
Since we cannot observe both $y_{0i}$ and $y_{1i}$ for the same individual, estimating ATE or ATT is challenging. A naive comparison between the treated and untreated groups results in selection bias:

\begin{equation}
E(y_{1i} \mid D_i = 1) - E(y_{0i} \mid D_i = 0)
\end{equation}

This difference consists of two terms:
\begin{equation}
E(y_{1i} \mid D_i = 1) - E(y_{0i} \mid D_i = 1) + E(y_{0i} \mid D_i = 1) - E(y_{0i} \mid D_i = 0)
\end{equation}

\begin{itemize}
    \item The first term represents the **true ATT**:
    \begin{equation}
    ATT = E(y_{1i} \mid D_i = 1) - E(y_{0i} \mid D_i = 1)
    \end{equation}
    \item The second term represents **selection bias**:
    \begin{equation}
    E(y_{0i} \mid D_i = 1) - E(y_{0i} \mid D_i = 0)
    \end{equation}
\end{itemize}

If selection bias is present, a direct comparison between treated and untreated individuals **does not accurately estimate ATT or ATE**.

\subsection{PSM Assumptions}

\subsection{Common Support Assumption}
For any possible value of $x_i$, the propensity score must satisfy:
\begin{equation}
0 < p(x_i) < 1
\end{equation}
This assumption ensures that there is **overlap between the treated and control groups**, making it possible to find comparable units.

\subsection{Balancing Assumption}
\begin{equation}
D_i \perp (y_{1i}, y_{0i}) \mid p(x_i)
\end{equation}
This assumption states that **conditional on the propensity score $p(x_i)$, treatment assignment is as good as random**. That is, for a given $p(x_i)$, there are no systematic differences between the treatment and control groups, meaning the treatment effect is entirely due to the treatment itself.

\textbf{1. Propensity Score Theorem}
The theorem states that if treatment assignment satisfies:
\begin{equation}
(y_0, y_1) \perp D \mid x
\end{equation}
then it also holds that:
\begin{equation}
(y_0, y_1) \perp D \mid p(x)
\end{equation}

\textbf{Proof:} Since $D$ is a binary variable, we need to show that:
\begin{equation}
P(D=1 \mid y_0, y_1, p(x))
\end{equation}
is independent of $y_0, y_1$. Using the expectation rule:
\begin{align*}
P(D=1 \mid y_0, y_1, p(x)) &= E[D \mid y_0, y_1, p(x)] \\
&= E_{y_0, y_1, x} [E(D \mid y_0, y_1, x) \mid y_0, y_1, p(x)] \quad \text{(by iterated expectation)} \\
&= E_{y_0, y_1, x} [E(D \mid x) \mid y_0, y_1, p(x)] \quad \text{(by ignorability assumption)} \\
&= E_{y_0, y_1, x} [p(x) \mid y_0, y_1, p(x)] \\
&= p(x)
\end{align*}
Since the probability only depends on $p(x)$, this confirms that $(y_0, y_1) \perp D \mid p(x)$.

\subsection{2. Overlap Assumption (Common Support)}
To ensure valid matching, every possible value of $x$ should exist in both treatment and control groups. This is known as the \textbf{overlap assumption} or \textbf{matching assumption}.

\textbf{Definition (Overlap Assumption):}
\begin{equation}
0 < p(x) < 1, \quad \forall x
\end{equation}
This ensures that there is a **common support** where treated and control groups have comparable observations.

\textbf{Implications:}
\begin{itemize}
    \item Ensures that treated and control groups share a common range of propensity scores.
    \item Helps avoid extrapolation in estimating treatment effects.
    \item Also called the \textbf{matching assumption}.
\end{itemize}

\textbf{Graphical Representation (Figure 28.1):}  
A graphical illustration shows the common support region, where both treated and control groups have overlapping propensity scores. Observations outside this range are often dropped to improve match quality.

\subsection{3. Steps for Estimating Treatment Effects Using Propensity Score Matching}
The general procedure for propensity score matching follows three steps:

\begin{enumerate}
    \item \textbf{Select Covariates $x_i$:}  
    Include all relevant variables that influence both treatment assignment $D$ and potential outcomes $(y_0, y_1)$. This ensures the \textbf{ignorability assumption} holds.

    \item \textbf{Estimate the Propensity Score $p(x)$:}  
    \textbf{Rosenbaum and Rubin (1985)} suggest using a logit model:
    \begin{equation}
    p(x) = Pr(D=1 \mid x)
    \end{equation}
    Including higher-order terms and interactions improves matching quality.

    \item \textbf{Perform Matching Based on $p(x)$:}  
    If the model is correctly specified, the \textbf{covariate distributions should be similar} between treated and control groups post-matching.

    \textbf{Data Balancing Check:}
    \begin{equation}
    \bar{x}_{\text{treat}} \approx \bar{x}_{\text{control}}
    \end{equation}
    Ensuring that the mean covariate values are similar across matched groups confirms that matching was successful.
\end{enumerate}


\end{document}
